\documentclass{article}

\usepackage{color}

\begin{document}

\title{\textcolor{red}{Up and Running with OpenGL in Windows}}
\author{Nazmus Saquib}
\maketitle

\section{Disclaimer}
The first thing I need to say is I don't support the approach of OpenGL that is being taught to us.
It is generally considered the ``legacy approach'', or some might even say ``deprecated''.
If you are someone who is into professional game development or highly intereseted in graphics,
take a look around the net - there are plenty of resources where they talk about modern approach of OpenGL.
As a bonus, if your target platform is the web i.e. building browser based games / graphics,
check out WebGL at your free time.

That being said, let's see how to setup OpenGL.

\section{Assumptions}
One misconception I've obsereved among people is that you need Visual Studio to code for OpenGL in windows.
That is not correct at all.
You can use pretty much any IDE you want, as long as you have a compiler you are golden.
Personally I use Vim as my text editor and MinGW compiler as the C/C++ compiler.
I would strongly suggest MinGW, as it comes packaged with the GNU make utility.
If you have CodeBlocks installed, chances are you already have minGW compiler.
But of course you are free to use any compiler you like.

I am going to assume the following:
\begin{enumerate}
\item You are a windows 32 bit user.
\item You have a C/C++ compiler installed.
\item The compiler's bin folder is in system's path.
\item I am going to term the compiler's home folder as $\langle$COMPILER\_HOME$\rangle$.
In my case this happens to be 
\begin{verbatim}
	C:\Program Files\CodeBlocks\MinGW
\end{verbatim}
\end{enumerate}

\section{Setup}
Inside the folder named \textbf{setup-files} you will find the necessary files.
Perform the following steps:
\begin{enumerate}
\item copy \textbf{glut.h} to \textbf{COMPILER\_HOME\textbackslash include\textbackslash GL}.
\item copy \textbf{libglut32.a} to \textbf{COMPILER\_HOME\textbackslash lib}.
\item copy \textbf{glut32.dll} to \textbf{C:\textbackslash Windows\textbackslash System32}.
Note that you might need to provide administrative clearance.
\end{enumerate}


\section{Verify Installation}
I have provided a c source code in the file \textbf{test.c}.
As different people will be using different IDE's, I can't really say how to configure them.
I personally prefer the command line.
Make sure you are in the folder containing the source code, launch cmd and type in the following:
\begin{verbatim}
	gcc -o test.exe test.c -lglut32 -lglu32 -lopengl32
\end{verbatim}
To run the program invoke (notice we leave the exe part):
\begin{verbatim}
.\test
\end{verbatim}
If you see a blue window, congrats - you've succesfully setup an OpenGL development environment!

To clarify the commands:
\begin{itemize}
\item \textbf{gcc}: I am using a gcc compiler, hence gcc. If the source code was in C++ I would have invoked g++.
If you are using a different compiler see its manual on how to invoke it.
\item \textbf{-o test.exe}: I want the output file (hence the -o) to have the name test.exe (hence it follows -o).
\item \textbf{test.c}: the name of the file containing source code.
\item \textbf{-lglut32 -lglu32 -lopengl32}: link (hence the -l) the glut32, glu32 and opengl32 libraries.
Note that they do not have to appear in any particular sequence.
Further note that we won't always need all these three libraries - but just to be safe better to keep these.
\end{itemize}

If all of this seems a lot of writing to you, just create a batch script or a makefile.
Personally I'm not a big fan of batch script, so I've added a sample makefile in the \textbf{setup-files} folder.
Note that the makefile might not work in your system if you don't have certain unix commands ported to windows.
Then again, if you are using an IDE you won't even need that.

\end{document}
